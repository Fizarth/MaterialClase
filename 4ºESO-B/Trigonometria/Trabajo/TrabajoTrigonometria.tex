\documentclass[a4paper,12pt]{extarticle}
\usepackage[margin=1.5cm]{geometry}
\usepackage{amsmath, amssymb}
\usepackage{array}
\usepackage{multicol}
\usepackage{helvet}
\usepackage{multirow}
\usepackage{graphicx}
\usepackage{hyperref} 


\begin{document}

    Habitualmente la enseñanza de la trigonometría se justifica mediante situaciones que se pueden modelar con triángulos rectángulos y en las que se necesita conocer la medida de algunos de sus lados, pero que no son accesibles para medir directamente, como alturas de montañas, pirámides o en Astronomía. 

    \subsubsection*{Guía de Fórmulas}

    Teorema de Tales: $\dfrac{\text{Altura del edificio}}{\text{Sombra del edificio}} =  \dfrac{\text{Altura del objeto}}{\text{Sombra del objeto}}$ 

    $tan(\alpha) = \dfrac{altura}{distancia}$

    \subsubsection*{Descripción del trabajo}
        En grupos de 3/4 alumnos deberemos calcular las dimensiones del instituto y presentar una memoria estructurada.



        \textbf{Fase 1: Preparación}
            \begin{itemize}
                \item Instalar app clinómetro: permite medir ángulos.
                \item Recordar las razones trigonométricas en triángulos rectángulos.
                \item Repasar la conversión de grados a radianes.
            \end{itemize}

        \textbf{Fase 2: Medidas}
            \begin{itemize}
                \item Elige un edificio alto del  entorno
                \item Medir la distancia horizontal del edificio al ángulo.
                \item Medir el ángulo de elevación
            \end{itemize}

        \textbf{Fase 3: Cálculo médiente distintos métodos}
            \begin{itemize}
                \item Aplicar la fórmula de la tangente para calcular la altura.
                \item Resolver el mismo problema utilizando seno y coseno.
                \item Comprobar la altura utilizando el Teorema de Tales.
                \item Comparar los resultados obtenidos.
            \end{itemize}

        \textbf{Fase 4: Trabajo en radianes}

            \begin{itemize}
                \item Convertir el ángulo medido a radianes.
                \item Expresar el valor en función de $\pi$.
                \item Repetir los cálculos utilizando la calculadora en modo radianes.
                \item Explicar qué representa un radián y por qué $\pi$ rad $= 180^\circ$.
            \end{itemize}

        \textbf{Fase 5: Justificación teórica}
            \begin{itemize}
                \item Dibujar el triángulo rectángulo asociado.
                \item Relacionar seno y coseno con la circunferencia goniométrica.
                \item Explicar por qué la tangente crece rápidamente cuando el ángulo se aproxima a $90^\circ$.
            \end{itemize}

        \textbf{Fase 6: Análisis del error}
            \begin{itemize}
                \item Comparar los resultados obtenidos por los distintos métodos.
                \item Calcular el error porcentual entre métodos.
                \item Analizar cómo afecta un posible error de $\pm 1^\circ$ en la medición del ángulo.
            \end{itemize}

        



    \subsubsection*{¿Qué hay que entregar?}

    \begin{enumerate}
        \item Datos del grupo (integrantes, fecha, objeto medido)
        \item Planteamiento del problema: Explicamos qué queremos calcular y por qué
        \item Datos que se han recogido
        \item Esquema gráfico (catetos, hipotenusas, ángulos)
        \item Desarrollo matemático
        \item Resultado final: ¿Altura del objeto:  ¿? Metros
        \item Reflexión final: ¿Ha sido preciso el método? ¿Qué posibles errores pueden aparecer? ¿Cómo mejoraríamos la medición?
    \end{enumerate}

    \subsubsection*{¿Qué se evaluará?}
    \begin{itemize}
        \item Uso correcto de las razones trigonométricas
        \item Claridad en la memoria
        \item Corrección de los cálculos
        \item Trabajo cooperativo 
        \item Presentación
    \end{itemize}



\end{document}