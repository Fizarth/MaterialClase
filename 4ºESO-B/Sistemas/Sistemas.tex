\documentclass[a4paper,12pt]{extarticle}
\usepackage[margin=1.5cm]{geometry}
\usepackage{amsmath, amssymb}
\usepackage{array}
\usepackage{multicol}
\usepackage{helvet}
\usepackage{multirow}
\usepackage{graphicx}
\usepackage{hyperref} 

\begin{document}

    \subsubsection*{1- Inecuaciones de segundo grado}
        \begin{enumerate}
            \item $x^2-9x+20<0$ (4,5)
            \item $(x+1) \cdot (x+4)\geq 0$ $(-\infty,-1] U [4,+\infty)$
            \item $4x^2-16x <0$ (0,4)
        \end{enumerate}
    \subsubsection*{1- Sistemas de ecuaciones lineales}

        \begin{enumerate}
            \item $\left \{\begin{array}{rr}
                x+2y=5 \\
                2x+y=7
            \end{array}\right .$  (x=3 y=1)

            \item $\left \{\begin{array}{rr}
                2x-3y=-25 \\
                4x-y=25
            \end{array}\right .$  (x=10 y=15)

            \item $\left \{\begin{array}{rr}
                2x-\dfrac{3x-y}{5}=\dfrac{22}{5} \\
                \dfrac{y}{3}+\dfrac{4x-3y}{4}=\dfrac{31}{12}
            \end{array}\right .$  (x=3 y=1)

            \item $\left \{\begin{array}{rr}
                2x+4y=7\\
                \dfrac{x}{3}-\dfrac{2x-5y}{6}=\dfrac{5}{4}
            \end{array}\right .$  (x=1/2 y=3/2)
        \end{enumerate}

    \subsubsection*{2- Sistemas de ecuaciones no lineales}

    \begin{enumerate}
        \item $\left \{\begin{array}{rr}
                y=x^2+4x-1\\
                y=2x+2
            \end{array}\right .$ (1,4) y (-3,-4)
        \item $\left \{\begin{array}{rr}
                x^2+y^2-4x-2y=20\\
                x^2+y^2-12x+2y=-12
            \end{array}\right .$ (6,4)(2,-4)
        \item $\left \{\begin{array}{rr}
                (x+3) \cdot y = -8\\
                x \cdot (y-1)=-3
            \end{array}\right .$ (1,2)(-9,4/3)
        \item $\left \{\begin{array}{rr}
                \dfrac{2}{x}+\dfrac{y+2}{xy}=0\\
                \dfrac{1}{x}+2y =-1
            \end{array}\right .$ (3, -2/3)
        \item $\left \{\begin{array}{rr}
                \sqrt{x+6}=y+1\\
                2x-y=-5
            \end{array}\right .$(-2,1) y una solución no válida.
    \end{enumerate}

    \subsubsection*{3- Sistemas de inecuaciones}

        \begin{enumerate}
            \item $\left \{\begin{array}{rr}
                5 \cdot (x+2) \leq x+2\\
                9 \cdot (x+1) \leq -4x+3 \cdot (x+1)
            \end{array}\right .$ (-$\infty$, -2)
            \item $\left \{\begin{array}{rr}
                5 \cdot (x+2) \leq x+2\\
                9 \cdot (x+1) \leq -4x+3 \cdot (x+1)
            \end{array}\right .$
            \item $\left \{\begin{array}{rr}
                5x-2 \leq 0\\
                3x+4 >4 \\
                \dfrac{x+9}{2} \geq 3
            \end{array}\right .$ (-4/3, 2/5)
            \item $\left \{\begin{array}{rr}
                z \geq 0\\
                2x+1 \geq 0 \\
                4x-3 <0
            \end{array}\right .$ (0,3/4)
        \end{enumerate}

    \subsubsection*{Problemas}
    \begin{enumerate}
        \item Halla dos números enteros cuya suma es 30 y su cociente es 4. (6 y 24)
        \item En una chocolatería hay 900 bombones envasados en cajas de 6 y 12 unidades.¿Cuántas cajas hay de cada clase si en total tienen 125 cajas?  (100 cajas de 6 y 25 cajas de 12)
        \item A un congreso acuden 60 personas. Si se van 3 hombres y vienen 3 mujeres, el número de mujeres sería un tercio del número de hombres. ¿Cuántos hombres y mujeres hay en el congreso? (48 hombres y 12 muejeres)
        \item Halla las edades de dos personas, sabiendo que hace 10 años la primera tenía 4 veces la edad de la segunda persona, pero dentro de 20 años la edad de la primera persona será el doble de la edad de la segunda. (70 años y 25 años)
        \item Por la mezcla de 400 kg de pienso de tipo A con 800 kg de pienso de tipo B se han pagado 2.200 €. Calcula el precio de cada tipo de pienso, sabiendo que, si se mezclase 1 kg de pienso de cada tipo, la mezcla costaría 3,90 €. (A=2.3€/kg y B = 1.60€/kg)
        \item Obtén un número de dos cifras cuya diferencia de sus cifras es 6 y la cifra de las unidades es el cuadrado de la cifra de las decenas. (39)
        \item Halla un número de dos cifras si el producto de sus cifras es 18 y la cifra de las unidades es el doble que la cifra de las decenas. (36)
        \item Halla dos números sabiendo que su suma es 16 y la suma de sus inversos es un tercio. (4 y 12)
    \end{enumerate}
\end{document}