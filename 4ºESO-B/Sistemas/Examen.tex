\documentclass[a4paper,12pt]{extarticle}
\usepackage[margin=1.5cm]{geometry}
\usepackage{amsmath, amssymb}
\usepackage{array}
\usepackage{multicol}
\usepackage{helvet}
\usepackage{multirow}
\usepackage{graphicx}
\renewcommand{\baselinestretch}{2}
\renewcommand{\familydefault}{\sfdefault}

\begin{document}

    \begin{tabular}{| c | c | c | c |}
        \hline
        & MATEMÁTICAS & 27/02/2026 & calificación \\
        \hline 

        \multirow{2}{*}{\includegraphics{../../img-Logos/iesoGalileo.jpg}} & Nombre y apellidos \hspace{120pt} & Ex. 4ºESO-D UD5:  & \\ 
        &&Sistemas de Ecuaciones& \\ 
        \hline
    \end{tabular}
    
    \begin{tabular}{| l | }
        \hline
        NOTAS: \\ 
            1. El examen tiene que ser limpio, ordenado y sin faltas de ortografía. \\
            
            2. El examen ha de realizarse en bolígrafo negro o azul, evitando tachones en la medida \\
            de lo posible. \\ 
            
            3. Deben aparecer todas las operaciones NO vale con indicar solo el resultado.\\
            
            4. Los problemas deben contener: Datos, planteamiento y resolución, respondiendo a lo \\
            que se pregunte, NO vale con indicar un número como solución del problema.\\
        \hline
    \end{tabular}

    \begin{enumerate}
        \item Resuelve los Sistemas
            \begin{enumerate}
                \item $\left \{\begin{array}{rr}
                4x-3y=23 \\
                2x+5y=-21
            \end{array}\right .$
            \end{enumerate}
        \item Determina dos números cuya suma es 5 y la suma de sus cuadrados es 13.
        \item Halla dos números sabiendo que su suma es 16 y la suma de sus inversos es un tercio.
        \item Por la mezcla de 400 kg de pienso de tipo A con 800 kg de pienso de tipo B se han pagado 2.200 €. Calcula el precio de cada tipo de pienso, sabiendo que, si se mezclase 1 kg de pienso de cada tipo, la mezcla costaría 3,90 €.
    \end{enumerate}
\end{document}