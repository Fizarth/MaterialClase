\documentclass[a4paper,12pt]{article}
\usepackage[margin=1.5cm]{geometry}
\usepackage{amsmath, amssymb}
\usepackage{array}
\usepackage{multicol}
\usepackage{graphicx}

\begin{document}

    \section{Límites y continuidad}

        \begin{itemize}
            \item Definición de límite y propiedades
                \[\lim_{x \to a}f(x)=L \]
                Cuando me acerco mucho al punto "a" en el eje de la X, la función se acerca mucho a L en el eje de la Y.

                Límites laterales:  A veces hay que distinguir si nos acercamos mucho por la derecha $\lim_{x \to a^+}f(x)$ o por la izquierda $\lim_{x \to a^-}f(x)$.

                El límite, si existe, es único y cumple que:
                \[\lim_{x \to a}f(x)=L \Leftrightarrow  \lim_{x \to a^+}f(x) = \lim_{x \to a^-}f(x)\]

                Límites al infinito: En un polinomio el término que manda es el de mayor grado

                Propiedades:
                
                Sean: $\lim_{x \to a}f(x)=A$ y $\lim_{x \to a}g(x) = B$
                \begin{enumerate}
                    \item $\lim_{x \to a}[k \cdot f(x)]=k \cdot A$
                    \item $\lim_{x \to a}[f(x) \pm g(x)]=A \pm B$
                    \item $\lim_{x \to a}[f(x) \cdot g(x)]=A \cdot B$
                    \item $\lim_{x \to a}\dfrac{f(x)}{g(x)} = \dfrac{A}{B}, B\neq 0$
                    \item $\lim_{x \to a}\sqrt[n]{f(x)} = \sqrt[n]{A}$, si existe $\sqrt[n]{A}$
                    \item $\lim_{x \to a}[f(x)^{g(x)} ]=A^B$, si $A^B$ es determinado
                \end{enumerate}

            \item Indeterminaciones
            
                Vamos a recordar primero algunas operaciones:

                \includegraphics[scale=0.9]{../img/Límites1.png}

                \begin{itemize}
                    \item \textbf{Indeterminación $\dfrac{0}{0}$}
                    
                        Polinomios: Simplificar la fracción (Factorizo polinomios)

                        Ejemplo: $\lim_{x \to 3}\dfrac{x^2-6-x}{x^2-3x} = \dfrac{0}{0}$ Indeterminación 
                        
                        $\lim_{x \to 3}\dfrac{(x-3)(x+2)}{x(x-3)} = \lim_{x \to 3} \dfrac{x+2}{x}= \dfrac{3+2}{3} = \dfrac{5}{3}$

                        Con raíces: Multiplicar el numerador y el denominador por el conjugado

                        Ejemplo: $\lim_{x \to 1}\dfrac{\sqrt{x+3}-2}{x^2-3x+2} = \dfrac{0}{0}$ Indeterminación 

                        $\lim_{x \to 1}\dfrac{(\sqrt{x+3}-2)(\sqrt{x+3}+2)}{(x^2-3x+2)(\sqrt{x+3}+2)} = \lim_{x \to 1}\dfrac{x+3-4}{(x^2-3x+2)(\sqrt{x+3}+2)} = \dfrac{0}{0} $ Indeterminación

                        $\lim_{x \to 1}\dfrac{(x-1)}{(x-1)(x-2)(\sqrt{x+3}+2)} = \dfrac{1}{-1 \cdot 4} = \dfrac{-1}{4} $

                    \item \textbf{Indeterminación $\dfrac{\infty}{\infty}$}
                    
                        Hay que dividir todos los términos por la x de mayor exponente.
                        
                        Ejemplo: $\lim_{x \to +\infty}\dfrac{x^2-2x}{2x^3-100} = \dfrac{\infty}{\infty}$ Indeterminación

                        $\lim_{x \to +\infty} \dfrac{\dfrac{x^2}{x^3}-\dfrac{2x}{x^3}}{\dfrac{2x^3}{x^3}-\dfrac{100}{x^3}} = \dfrac{0-0}{2-0} = \dfrac{0}{2} = 0 $

                        Ejemplo: $\lim_{x \to +\infty}\dfrac{\sqrt{4x^2-3}-3x}{x-3} = \dfrac{\infty}{\infty}$ Indeterminación.

                        $\lim_{x \to +\infty}\dfrac{\dfrac{\sqrt{4x^2-3}}{x}+\dfrac{3x}{x}}{\dfrac{x}{x}-\dfrac{3}{x}} = \dfrac{\sqrt{4}+3}{1} = 5$

                    \item \textbf{Indeterminación $\infty - \infty$}
                    
                        Aparecen raíces: Multiplicar y dividir por el conjugado
                        Sin raíces: Realizar la operación que aparece.

                        Ejemplo: $\lim_{x \to +\infty}(\sqrt{x-1}-\sqrt{x}) = \infty - \infty$ Indeterminación

                        $\lim_{x \to +\infty} \dfrac{(\sqrt{x-1}-\sqrt{x})\cdot (\sqrt{x-1}+\sqrt{x})}{(\sqrt{x-1}+\sqrt{x})} = \lim_{x \to +\infty} \dfrac{x-1-x}{(\sqrt{x-1}+\sqrt{x})}  =$
                        
                        $ \lim_{x \to +\infty} \dfrac{-1}{(\sqrt{x-1}+\sqrt{x})} = \dfrac{-1}{\infty} $
                \end{itemize}

            \item Continuidad de una función
            
                Cuando soy capaz de repasar el trazo sin levantar el lápiz del papel.

                \includegraphics[scale=0.5]{../img/Límites2.png}

                Una función f(x) es continua en un punto x=a si cumple:

                \begin{enumerate}
                    \item $\exists f(a) \rightarrow $ Al sustituir x=a en la función, me da un valor real.
                    \item $\exists \lim_{x \to a} f(x) \rightarrow$ Existen los límites laterales
                    \item $f(a)=\lim_{x \to a} f(x) $
                \end{enumerate}

                f(x) continua en x=a $\Leftrightarrow$ f(a) = $\lim_{x \to a} f(x) =L$

                En caso contrario, la función será discontinua (o no será continua) en x=a.

                ¡Recuerda! Son continuas las funciones:

                \begin{itemize}
                    \item Polinómicas
                    \item Exponenciales
                    \item Logarítmicas
                    \item Racionales: Menos los valores que anulan el denominador
                    \item Radicales: Solo donde el radicando es $\geq 0$
                \end{itemize}


                Funciones definidas a trozos: $f(x) = \begin{cases}
                            1 & \text{si } x < 0\\
                            x+1  & \text{si } 0<x<1\\
                            x^2-2x  & \text{si } x \geq 1.
                            \end{cases}$
                    \begin{enumerate}
                        \item Ver si las funciones definidas en sus "trozos" son o no son continuas.
                        \begin{enumerate}
                            \item continúa, ya que sus "trozos" son funciones continúa (polinomios)
                        \end{enumerate}
                        \item Estudiar la continuidad en los valores que la función se divide a trozos.
                        \begin{enumerate}
                            \item x=0: no es continua 
                            \item x=1 no es continua
                        \end{enumerate}
                        \item Agrupamos las continuidades del paso 1 y 2
                        \begin{itemize}
                            \item f(x) es continua $\forall x \in \mathbb{R}-\{0,1\}$
                        \end{itemize}
                    \end{enumerate}

                Tipos de discontinuidad: 

                \begin{itemize}
                    \item Discontinuidad evitable
                    \item Discontinuidad inevitable
                    \begin{itemize}
                        \item salto finito
                        \item salto infinito
                    \end{itemize}
                \end{itemize}

                
            \item Asíntotas
            
                Son rectas a las que se pega la función. Hay tres tipos: verticales, horizontales y oblicuas.

                \begin{itemize}
                    \item Asíntotas verticales: $f(x)$ tiene A.V. en x=a $\Leftrightarrow$ $\lim_{x \to a}f(x) = \pm \infty$
                        \begin{enumerate}
                            \item Buscar los puntos que no están en el dominio
                            \item Calcular $\lim_{x \to a}f(x)$, si sale un número no tiene asíntota vertical, si sale $\pm  \infty$ tiene asíntota vertical en x=a
                        \end{enumerate}
                    \item Asíntota horizontal: $f(x)$ tiene A.H. en y =a $\Leftrightarrow$ $\lim_{x \to \pm \infty}f(x)=a$
                        \begin{enumerate}
                            \item Calcular $\lim_{x \to \pm \infty}f(x)$ Si da $\pm \infty$ no tiene A.H. si da un número (a), tiene asíntota horizontal en y =a 
                        \end{enumerate}
                    \item Asíntota oblicua: $y = mx + n $ \\
                        El grado del numerador - grado de denominador = 1
                        \begin{itemize}
                            \item División de polinomios: El cociente es la asíntota
                            \item $m = \lim_{x \to \infty}\dfrac{f(x)}{x}$ y $n=\lim_{x \to \infty}(f(x))-mx$
                        \end{itemize}

                    IMPORTANTE: Si hay A. Horizontal no hay A. Oblicua.
                \end{itemize}
        \end{itemize}

    \section{Ejercicios}

    \begin{enumerate}
        \item Calcula a
            \begin{multicols}{2}
                \begin{enumerate}
                    \item $\lim_{x \to 3}\dfrac{ax^2-9a}{x^2-2x-3}=1$
                    \item $\lim_{x \to \infty}(\sqrt{x^2+ax+1}-x)=2$
                \end{enumerate}
            \end{multicols}
        \item Indica los puntos de discontinuidad de cada una de las siguientes funciones. 
            \begin{multicols}{3}
                \begin{enumerate}
                    \item $f(x)=x^3+8$
                    \item $f(x)=\dfrac{x}{x^3+8}$
                    \item $f(x)=\dfrac{x}{x^2-8}$
                    \item $f(x)=\dfrac{x}{x^2+8}$
                    \item $f(x)=\sqrt{x^3-8}$
                    \item $f(x)=\sqrt{x^4}$
                    \item $f(x)=\dfrac{x}{\sqrt{x^2+4}}$
                    \item $f(x)=\dfrac{x}{\sqrt{x^2-2x}}$
                    \item $f(x)=e^{x-2}$
                    \item $f(x)=e^{\dfrac{1}{x}}$
                    \item $f(x)=log(5x-6)$
                    \item $f(x)=\dfrac{1}{x^2+2}$
                    \item $f(x)=tan(2x)$
                    \item $f(x)=sin \dfrac{1}{x-1}$
                    \item $f(x)=cos(2x-1)$
                    \item $f(x)= \dfrac{2-sin x}{2+ cos x}$
                \end{enumerate}
            \end{multicols}
        \item Estudia la continuidad y di que tipo de discontinuidad en cada caso.
            \begin{multicols}{2}
                \begin{enumerate}
                    \item $f(x) = \begin{cases}
                            3x-x^2 & \text{si } x \leq 3\\
                            x-3  & \text{si } 3<x<6\\
                            0  & \text{si } x \geq 6.
                            \end{cases}$ %Continua en todos menos el 6

                    \item $f(x) = \begin{cases}
                            e^{1-x^2} & \text{si } x \leq -1\\
                            \dfrac{-1}{x}  & \text{si } x>-1
                            \end{cases}$ %continua en todos menos el 0
                    \item $f(x) = \begin{cases}
                            x^2   & \text{si } x \leq 1\\
                            x-1  & \text{si } x>1
                            \end{cases}$ 
                    \item $f(x) = \begin{cases}
                            1-x^2   & \text{si } x \leq 1\\
                            x-1  & \text{si } x>1
                            \end{cases}$
                    \item $f(x) = \begin{cases}
                            cos(x)   & \text{si } x \leq 0\\
                            x+1  & \text{si } x>0
                            \end{cases}$
                    \item $f(x) = \begin{cases}
                            \dfrac{1}{x-1}   & \text{si } x \leq 0\\
                            sin(x)  & \text{si } x>0
                            \end{cases}$
                    \item $f(x) = \begin{cases}
                            e^{x-1}   & \text{si } x \leq 1\\
                            \dfrac{-1}{x-2}  & \text{si } x>1
                            \end{cases}$
                    \item $f(x) = \begin{cases}
                            1-x   & \text{si } x \leq 1\\
                            ln(2-x)  & \text{si } x>1
                            \end{cases}$

                \end{enumerate} 
            \end{multicols}
        \item Halla el valor de a y b para que la función f(x) sea continúa.
                \begin{multicols}{2}
                    \begin{enumerate}
                        \item $f(x) = \begin{cases}
                            x^2+x & \text{si } x < -1\\
                            3x-a  & \text{si } x\geq -1
                            \end{cases}$
                        
                        \item $f(x) = \begin{cases}
                            e^x & \text{si } x \leq 0\\
                            2x+a  & \text{si } x>0
                            \end{cases}$
                        \item $f(x) = \begin{cases}
                            x-a & \text{si } x < -2\\
                            x^2 +x+b  & \text{si } -2 \leq x <1 \\
                            \dfrac{1}{x+1} & \text{si }x \geq 1
                            \end{cases}$
                        \item $f(x) = \begin{cases}
                            x^2 & \text{si } x \leq 1\\
                            2x+a  & \text{si } x<1
                            \end{cases}$
                        \item $f(x) = \begin{cases}
                            e^{ax} & \text{si } x \leq 0\\
                            2a+sin(x)  & \text{si } x>0
                            \end{cases}$
                        \item $f(x) = \begin{cases}
                            a+ln(1-x) & \text{si } x <0\\
                            x^2e^{-x}  & \text{si } x\geq 0
                            \end{cases}$
                    \end{enumerate}
                \end{multicols}
        \item Halla para qué valores de "a" la función  $f(x) = \begin{cases}
                            x^2 & \text{si } x \leq a\\
                            a+2  & \text{si } x>a
                            \end{cases}$

    \end{enumerate}

    \section{Videos de refuerzo}

        \begin{itemize}
            \item Ejercicios de límites: https://www.youtube.com/watch?v=YnWo7noYDyQ
            \item Tipos de discontinuidad: https://www.youtube.com/watch?v=raDNa4Q8QkQ
            \item Ejercicios de continuidad: https://www.youtube.com/watch?v=7ZYPMfwp47k
            \item Ejercicios de continuidad: https://www.youtube.com/watch?v=uIyliInYZek
            \item Indeterminación $\dfrac{0}{0}$ : https://www.youtube.com/watch?v=MKjHcw3ooUc
            \item Indeterminación $\dfrac{0}{0}$: https://www.youtube.com/watch?v=HFnvI-ZTm2E
            \item Indeterminación $\dfrac{\infty}{\infty}$: https://www.youtube.com/watch?v=y26Uv5jpvWc
            \item Indeterminación $\infty - \infty$: https://www.youtube.com/watch?v=OFA5P-FrqvY
            \item Regla de l'Hopital : https://www.youtube.com/watch?v=iCjRT0lYqPE
            \item Lista de reproducción: https://www.youtube.com/playlist?list=PLwCiNw1sXMSDRxTynmzWuH4wVZSFmSdv7
        \end{itemize}


\end{document}
