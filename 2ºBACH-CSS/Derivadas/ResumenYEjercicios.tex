\documentclass[a4paper,12pt]{article}
\usepackage[margin=1.5cm]{geometry}
\usepackage{amsmath, amssymb}
\usepackage{array}
\usepackage{multicol}
\usepackage{graphicx}
\usepackage{hyperref}

\begin{document}
    \section*{Derivadas}

    \subsection*{Reglas de derivación}
    
        \includegraphics{../img/Derivadas2.png}

    \section*{Ejercicios}
    \begin{enumerate}
        \item Deriva las siguientes funciones y simplifica:
            \begin{enumerate}
                \item $f(x) = e^{x^2-2}\sqrt{x+1}$
                \item $g(x) = (ln(3x+2))^3$
                \item $h(x) = \dfrac{1}{\sqrt{3x-1}}$
            \end{enumerate}
        \item Calcula para qué valor o valores de a $f(x) = \begin{cases}
                            3-ax^2 & \text{si } x \leq 1\\
                           \dfrac{2}{ax}  & \text{si } x>1
                            \end{cases}$ es derivable en x=1. Representa la función para a=3
        \item Dada la función $f(x) = e^{\dfrac{2x}{1+x^2}}$, calcula  sus extremos relativos (máximos y mínimos) e intervalos de crecimiento  y decrecimiento.
        \item Estudia la monotonía  de la función $f(x) = (x+1)e^x$
        \item Dada la función $f(x)=mx^3+nx+1$
            \begin{enumerate}
                \item Halla m y n sabiendo que la función pasa por el punto (1,2) y la pendiente de su recta tangente en x=1 es 3. 
                \item Para m=1 y n=0 calcula la ecuación de la recta tangente a la función PARALELA a la recta y-x=2
            \end{enumerate}
        \item Sabiendo que la función $f(x) = x^3+ax^2+bx+c$ corta al eje de abscisas en x=-1 y que tiene un extremo relativo en el punto (2,1), halla el valor  de los parámetros a, b y c. 
        \item Supongamos que el consumo eléctrico de un país entre las 0 y las 8 horas viene dado por la función $C(x) = 10x-x^2+16$ (expresado en gigavatios), con $0 \leq x  \leq 8$. Determínese cuáles son el consumo máximo y el mínimo en este intervalo de tiempo y cuando se alcanzan.
        \item Descomponer el número 36 en dos sumandos positivos de modo que el producto del primer sumando por el cuadrado del segundo sea máximo. 
        




    \end{enumerate}
    \section*{Videos}
    \begin{itemize}
        \item Operaciones  con derivadas  I: https://www.youtube.com/watch?v=s5QJAEYwgKU
        \item Ejercicios  de probabilidad II: \url{https://www.youtube.com/watch?v=Fyo4R_-qPAg}
        \item Recta tangente a una función: https://www.youtube.com/watch?v=7tU2EZdVlmo
        \item Ecuación recta tangente: https://www.youtube.com/watch?v=jZVaJFw3y3g
        \item Ecuación recta tangente: https://www.youtube.com/watch?v=xtBHnChjiqc
    \end{itemize}
\end{document}