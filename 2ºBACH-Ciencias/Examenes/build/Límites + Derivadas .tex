
\documentclass[14pt]{extarticle}
\usepackage[margin=1.5cm]{geometry}
\usepackage{amsmath, amssymb}
\usepackage{array}
\usepackage{multicol}
\usepackage{graphicx}

\begin{document}

    Matemáticas II : Examen de Límites y Derivadas.

    \begin{enumerate}

        \item Calcula los siguientes límites: 
            \begin{enumerate}
                \item $\lim_{x \to \infty } ( (\sqrt{x^2  +5})-(x-2) )$
                \item $\lim_{x \to 0 }  \dfrac{x^2(1-2x)}{x-2x^2 - sen (x)}$
            \end{enumerate} 


        \item  Dada la función $f(x)= x^2 \cdot e^{-x}$
        \begin{enumerate}
            \item Calcula los intervalos de crecimiento  y decrecimiento
            \item Calcula $\lim_{x  \to \infty} f(x)$
        \end{enumerate}


        
        \item Calcula los valores de $a$ y $b$ para que la función sea continua y derivable.
        
            $f(x) = \begin{cases}
                            x^3 +ax^2  & \text{si } x < 1\\
                            bx+ ln (x)  & \text{si } x \geq 1
                            \end{cases}$ 
                            
        \item Considera la funcioón $f(x) = \dfrac{x^2x}{e^x}$ 
        \begin{enumerate}
            \item Calcule los extremos relativos.
            \item Intervalos de crecimiento y decrecimiento.
        \end{enumerate}
        \item La hipotenusa de un triángulo rectángulo mide 1dm. 
        
        Hacemos girar el triángulo al redor de uno de sus catetos. 
        
        Determina la longitud de los catetos, de forma que el cono  engendrado de esta forma  tenga un volumen máximo. 
        
        Volumen del cono: $V_C = \dfrac{\pi r^2 h}{3}$

        \includegraphics [scale=0.7]{../img/cono.png}
        
    \end{enumerate}

\end{document}
