\documentclass[a4paper,12pt]{article}
\usepackage[margin=1.5cm]{geometry}
\usepackage{amsmath, amssymb}
\usepackage{array}
\usepackage{multicol}

\begin{document}
\section*{Ecuaciones}
    Una \textbf{ecuación} es una propuesta de igualdad que se cumple solo para algunos valores de la incógnita, y en algunos casos para ninguno.

    Las \textbf{soluciones} de una ecuación son los valores que hacen cierta la igualdad.

    \textbf{Resolver} una ecuación es hallar su solución (o soluciones) o llegar a la conclusión de que no tiene.

    \subsection*{Ecuaciones de primer grado} 
        Una ecuación de primer grado es aquella que solo aparecen expresiones algebraicas de grado 1. Después de simplificarla llegaremos a una igualdad tipo: $ax+b=0$ 

        Pasos: 
        \begin{enumerate}
            \item Quitar denominadores
            \item Quitar paréntesis
            \item Reducir cada miembro 
            \item Despejar la incógnita
            \item Comprobar si el valor obtenido cumple la igualdad.
        \end{enumerate}

        Ejemplo: $\dfrac{3(x+1)}{4}-2=2x-\dfrac{5(x-1)}{3}$
        \begin{enumerate}
            \item mcm(4,3) = 12
            
            $3 \cdot 3(x+1) -24 =24x - 4 \cdot 5 (x-1)$
            \item $9x+9-24=24x-20x+20$
            \item $9x-15=4x+20$
            
            $9x-4x = 20+15$

            $5x=35$
            \item $x=\dfrac{35}{5}$
            
            $x=7$
            \item $\dfrac{3(7+1)}{4}-2=\dfrac{3 \cdot 8}{4}-2  =6-2=4$
            
            $2 \cdot n 7-\dfrac{7(7-1)}{3} = 14 - \dfrac{5 \cdot 6}{3}=14-10=4$
        \end{enumerate}

        Existen expresiones como: 
         \begin{itemize}
            \item $4x-6=4(x+3) \rightarrow 4x-6 = 4x+12 \rightarrow 0 \cdot x = 18$ La cual \textbf{no} tiene solución. 
            \item $4x-6=4(x-2)+2 \rightarrow 4x-6=4x-6 \rightarrow 0 \cdot x = 0$ La cual tiene \textbf{infinitas} soluciones.
         \end{itemize}

    \subsection*{Ecuaciones de segundo grado}

         Las ecuaciones de segundo grado son de la forma: $ax^2 +bx +c = 0$ y se resuelven con la siguiente formula : $ x= \dfrac{-b  \pm \sqrt{b^2-4ac}}{2a}$ 
         \begin{itemize}
            \item Si $b^2-4ac >0$ hay dos soluciones.
            \item Si $b^2-4ac =0$ hay una solución.
            \item $b^2-4ac <0$ no hay ninguna solución.
         \end{itemize}

         Ejemplo: $5x^2+3x-2=0 \rightarrow x=\dfrac{-3 \pm \sqrt{3^2-4 \cdot 5  \cdot (-2)}}{2 \cdot 5} = \dfrac{-3 \pm  \sqrt{9+40}}{10} \rightarrow x_1 = \dfrac{2}{5}$ y $  x_2 = -1$ 

         \subsubsection*{Ecuaciones incompletas}
         \begin{itemize}
            \item Si $b=0$ despejamos directamente $x^2$

                Ejemplo: $3x^2-48=0 \rightarrow 3x^2 = 48 \rightarrow x^2 = 16 \rightarrow x= \pm \sqrt{16} \rightarrow x = \pm 4$
            \item Si $c=0$ factorizamos sacando factor común.
            
                Ejemplo: $2x^2-x=0 \rightarrow x(2x-1)=0 \rightarrow x_1 = 0$ y $ 2x-1 =0 \rightarrow x_2= \dfrac{1}{2}$
         \end{itemize}

        \subsubsection*{Pasos para resolver una ecuación de segundo grado}.
            \begin{enumerate}
                \item Quitar los denominadores, si los hay.
                \item Quitar paréntesis, si los hay. 
                \item Reducir a la forma $ax+bx+c=0$ 
                \item Resolver con la fórmula o con el resto de recursos que conoces.
            \end{enumerate}

    \subsection*{Otros tipos de ecuaciones}
        \subsubsection*{Ecuaciones factorizadas}

            Aparece como producto de factores y para que el producto sea cero es necesario que uno de los factores sea cero.

            Ejemplo: $x(x-1)(x^2-5x+6)=0$
            \begin{itemize}
                \item $x_1 =0$
                \item $x-1 = 0 \rightarrow x_2 = 1$
                \item $x^2-5x+6=0 \rightarrow x_3 = 2$ y $x_4=3$
                
                La ecuación tiene 4 soluciones.
            \end{itemize}
        \subsubsection*{Ecuaciones con x en el denominador}

            Habrá que suprimir los denominadores.

            Ejemplo: $\dfrac{8}{x}-3 = \dfrac{5}{x+3}$ 
            \begin{itemize}
                \item Para suprimir los denominadores, haciendo el mínimo común múltiplo.
                
                $8 \cdot (x+3) - 3 \cdot x \cdot (x+3) = 5x$

                $-3x^2-6x+24=0$

                $x^2+2x-8=0$

                $x=\dfrac{-2 \pm \sqrt{4+32}}{2} = \dfrac{-2 \pm 6}{2} \rightarrow x_1 = 2$ y $x_2=-4$
                \item Es necesario comprobar las soluciones. 
                
                $\dfrac{8}{2}-3=\dfrac{5}{2+3}$

                $\dfrac{8}{-4}-3=\dfrac{5}{-4+3}$

                Por lo tanto, tiene dos soluciones.
            \end{itemize}
        \subsubsection*{Ecuaciones con radicales}
            Resolvamos la ecuación: $\sqrt{x^2+5}+1=2x$
            \begin{enumerate}
                \item Aislamos el radical en un miembro, pasando al otro lado lo demás.
                
                $\sqrt{x^2+5}=2x-1$
                \item Elevamos al cuadrado los dos miembros.
                
                $(\sqrt{x^2+5})^2 = (2x-1)^2$

                $x^2+5=4x^2-4x+1$ 
                \item Reducimos la ecuación a la forma general y la resolvemos.
                
                $3x^2-4x-4=0$ 

                $x=\dfrac{4 \pm \sqrt{16+48}}{6} = \dfrac{4 \pm 8}{6} \rightarrow x_1=2$ y $x_2=-2/3$

                \item Será necesario comprobar las soluciones, ya que, al elevar al cuadrado pueden aparecer soluciones que en realidad no son.
                
                $\sqrt{2^2+5}+1 = 2 \cdot 2$ 

                $\sqrt{(-2/3)^2+5}+1  \neq 2(-2/3)$ 

                En este casa x=2 es solución, pero $x=\dfrac{-2}{3}$ no lo es.
            \end{enumerate}
\newpage
\section*{Ejercicios}
    \begin{enumerate}
        \item Resolver las siguientes ecuaciones de primer grado
            \begin{enumerate}
                \item $5x-4(2x+3)=5(x-1)-8x$ (no tiene solución) $\vspace{3cm}$
                \item $7+6(3x-2)=15x-(5-3x)$ (da igual el valor de x) $\vspace{3cm}$
                \item $\dfrac{3-x}{2}-\dfrac{2(x-2)}{3}=4-\dfrac{7(2x-1)}{9}$ $\vspace{3cm}$
            \end{enumerate}         
        \item El sueldo de un cajero del supermercado, aumentado en su tercera parte y 80€, se iguala con el de su encargada, que gana 1700€ ¿Cuánto gana el cajero? (1215€) $\vspace{5cm}$
        \item Una porción de pizza cuesta 40 céntimos más  que un bote de refresco. Por tres botes de refresco y cuatro porciones de pizza, hemos pagado 10€. ¿Cuánto cuesta cada bote y cuánto cada porción? (refresco 1.20€ y pizza 1.60€)$\vspace{5cm}$
        \item Entre mi hermano y yo tenemos ahorrados 248€, y si yo le diera 13 ambos tendríamos lo mismo. ¿Cuánto tenemos cada uno? $\vspace{5cm}$
        \item Una bodega tiene dos tipos de vino, uno de calidad superior, a 5€/L, y otro de calidad inferior, a 3.60€/L ¿Cuántos litros del primero debe añadir a un barril que contiene 100 litros del segundo, para que la mezcla resulte a 4€/L? $\vspace{5cm}$
        \item Una camisa y una falda costaban lo mismo antes de las rebajas. Ahora he comprado ambas prendas por 99€, con una rebaja del 20\% en la camisa y 15\% en la falda. ¿Cuánto costaba cada prenda antes de la rebaja? $\vspace{5cm}$
        \item Multiplicando la edad de Laura por la que tenía el año pasado, obtenemos 240. ¿Cuantos años tiene Laura? (16 años)$\vspace{5cm}$
        \item Una escalera está apoyada en la pared, alcanzando una altura igual a 2.4 veces la separación entre su apoyo inferior y la pared, y 20 cm menor que la propia longitud de la escalera. ¿Cuánto mide la escalera? (2.60m) $\vspace{5cm}$
        \item Una finca rectangular con una superficie de $28hm^2$ está rodeada por una valla de 22hm de longitud. ¿Cuáles son las dimensiones de la finca? (7hm y 4hm) $\vspace{5cm}$
        \item Un inversor deposita en el banco 10.000€ a un cierto porcentaje. Al cabo de un año, añade 20.000€ y mantiene todo el capital al mismo porcentaje. Al finalizar el segundo año le devuelven 32.025€. ¿A qué porcentaje impuso su capital? (5\%)$\vspace{5cm}$
        \item Ecuaciones: 
            \begin{enumerate}
                \item $x(x-1)(x^2-5x+6)=0$ (x= 0,1,2,3) $\vspace{5cm}$
                \item $\dfrac{8}{x}-3=\dfrac{5}{x+3}$ (x=2,-4) $\vspace{5cm}$
                \item $\sqrt{x^2+5}+1=2x$ (x=2) $\vspace{5cm}$
                \item $9(2-3x)+\dfrac{4}{5}(x-3)=4x-\dfrac{7-3x}{5}$ (x=85/154) $\vspace{5cm}$
                \item $6-(8-4(3x-\dfrac{3}{7}))=2x-\dfrac{5-9x}{7}$ (x=21/61) $\vspace{5cm}$
                \item $\dfrac{1-2x^2}{3x}-\dfrac{2}{5} = \dfrac{4x-2}{15}$ (x=0.47 o x=-0.75) $\vspace{5cm}$
                
            \end{enumerate} 
        \item Comprueba que la solución de $\dfrac{x-1}{2}-\dfrac{x+1}{3}=\dfrac{1}{6}$ es x=6 $\vspace{5cm}$
        \item Resuelve las ecuaciones de segundo grado incompletas.
        \begin{multicols}{2}
            \begin{enumerate}
                \item $^2+6x=0$ (x=0 y x=-6) $\vspace{3cm}$
                \item $3x^2+18x=0$ (x=0 y x=-6) 
                \item $5x^2-180=0$ (x=6, x=-6) $\vspace{3cm}$
                \item $5x^2-10x=0$ (x=0, x=2) 
            \end{enumerate}
            
        \end{multicols}$\vspace{3cm}$

        \item Resuelve las ecuaciones.
        \begin{enumerate}
            \item $x(x^2-64)=0$ $\vspace{5cm}$
            \item $(x+1)(x^2-4)=0$ $\vspace{5cm}$
            \item $(2x+1)(x^2+5x-24)=0$ $\vspace{5cm}$
            \item $(x-4)(\dfrac{4}{3x-1}-2)=0$ $\vspace{5cm}$
        \end{enumerate}
        \item Elimina los denominadores y resuelve.
            \begin{enumerate}
                \item $\dfrac{12}{x}+1=x+2$  $\vspace{5cm}$
                \item $\dfrac{7}{x}-2=x+\dfrac{4}{x}$ $\vspace{5cm}$
                \item $\dfrac{5}{x^2+1}+1=\dfrac{10}{x^2+1}$ $\vspace{5cm}$
                \item $\dfrac{2}{3x-1}+x=\dfrac{x+3}{3x-1}$ $\vspace{5cm}$
                \item $\dfrac{5}{x-3}-1=x$ $\vspace{5cm}$
                \item $\dfrac{8}{x}-3=\dfrac{5}{x+3}$ $\vspace{5cm}$
                \item $\dfrac{15}{x-1}=\dfrac{12}{x}+1$ $\vspace{5cm}$
                \item $\dfrac{7}{x+2}+2 =\dfrac{9}{x-2}$ $\vspace{5cm}$
            \end{enumerate}
        \item Resuelve las ecuaciones con radicales.
            \begin{enumerate}
                \item $\sqrt{x}-3=0$ $\vspace{5cm}$
                \item $\sqrt{x}+2=x$ $\vspace{5cm}$
                \item $\sqrt{4x+5}=x+2$ $\vspace{5cm}$
                \item $\sqrt{x+1}-3=x-8$ $\vspace{5cm}$
                \item $3\sqrt{x-1}=2x-11$ $\vspace{5cm}$
                \item $x=\sqrt{2x^2-1}$ $\vspace{5cm}$
                \item $\sqrt{2x^2-2}=1-x$ $\vspace{5cm}$
                \item $\sqrt{3x^2+4}=\sqrt{5x+6}$ $\vspace{5cm}$
            \end{enumerate}
        \item Calcula las dimensiones de un rectángulo sabiendo que su perímetro es 30cm y que su base es doble que su altura. (altura 5cm y base 10 cm) $\vspace{5cm}$
        \item Con motivo de la jubilación de un empleado, varios compañeros deciden regalarle, como recuerdo, un reloj que cuesta 240€. Al conocer la idea se apuntan otros cuatro más y, así tocan todos a 10€ menos. ¿Cuántos participan finalmente en la compra de regalo? $\vspace{5cm}$
        \item Para salvar el desnivel de un metro, se ha construido una rampa que es 10cm más larga que su proyección horizontal. ¿Cuál es la longitud de la rampa?$\vspace{5cm}$
        \item Una comerciante de un mercadillo ha obtenido 240€ por la venta de cierta cantidad de camisas. Habría obtenido lo mismo vendiendo 6 unidades menos, pero dos euros más caras. ¿Cuántas camisas ha vendido? $\vspace{5cm}$
        
            
       
    \end{enumerate}

    \section*{Videos}
    \begin{itemize}
        \item Común denominador de fracciones: https://www.youtube.com/watch?v=x2pARYoIXCU
        \item 
    \end{itemize}
\end{document}