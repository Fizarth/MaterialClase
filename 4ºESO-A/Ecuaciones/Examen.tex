\documentclass[a4paper,12pt]{extarticle}
\usepackage[margin=1.5cm]{geometry}
\usepackage{amsmath, amssymb}
\usepackage{array}
\usepackage{multicol}
\usepackage{helvet}
\usepackage{multirow}
\usepackage{graphicx}
\renewcommand{\baselinestretch}{2}
\renewcommand{\familydefault}{\sfdefault}

\begin{document}
\begin{tabular}{| c | c | c | c |}
    \hline
    & MATEMÁTICAS & 19/12/2025 & calificación \\
    \hline 

    \multirow{2}{*}{\includegraphics{../../img-Logos/IesJosePlanes.png}} & Nombre y apellidos \hspace{120pt} & Ex. 4ºESO-D UD5:  & \\ 
    &&Ecuaciones& \\ 
    \hline
\end{tabular}

\begin{tabular}{| l | }
    \hline
    NOTAS: \\ 
        1. El examen tiene que ser limpio, ordenado y sin faltas de ortografía. \\
        
        2. El examen ha de realizarse en bolígrafo negro o azul, evitando tachones en la medida \\
         de lo posible. \\ 
        
        3. Deben aparecer todas las operaciones NO vale con indicar solo el resultado.\\
        
        4. Los problemas deben contener: Datos, planteamiento y resolución, respondiendo a lo \\
        que se pregunte, NO vale con indicar un número como solución del problema.\\
    \hline
\end{tabular}

$\vspace{0.5cm}$

FÓRMULAS:
\begin{itemize}
    \item $(a+b)^2 = a^2 + b^2 +2ab$ 
    \item $(a-b)^2 = a^2 + b^2 -2ab$ 
    \item $(a+b)(a-b) = a^2 - b^2$ 
\end{itemize}

\begin{enumerate}
    \item (1.5 puntos) Resuelve la ecuación.
    \[ \dfrac{3(x+2)}{2}+\dfrac{x-1}{5} = \dfrac{2(x+1)}{5}+\dfrac{37}{10} \] % x=1
    $\vspace{5cm}$

    \newpage
    \item (1.5 puntos) Resuelve la ecuación.
    \[15 - (x + 2)^2 = (x - 3)^2 + 2x\]
    $\vspace{5cm}$

    \item (1.5 puntos) Resuelve la ecuación.
    \[\sqrt{x+1}-3=x-8\] %8 si, 3 no
    $\vspace{5cm}$

    \item (1.5 puntos) Resuelve la ecuación.
    \[ \dfrac{2}{3x-1}+x=\dfrac{x+3}{3x-1}\]
    $\vspace{5cm}$
    
    \newpage
    \item (2 puntos) Lucía tiene cinco veces el dinero de Paula. Si Lucía le diera 800 € a Paula, entonces Lucía tendría el triple que Paula. ¿Cuánto dinero tiene cada una? %(Paula 1600, Lucia 8000)
    $\vspace{7cm}$

    
    \item (2 puntos) Si multiplico mi edad por la que tenía el año pasado, obtengo el mismo resultado que si multiplico la que tenía hace cuatro años por la que tendré dentro de cuatro. ¿Cuántos años tengo? %16
    $\vspace{5cm}$
\end{enumerate}
\end{document}