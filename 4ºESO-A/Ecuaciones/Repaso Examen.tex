\documentclass[a4paper,12pt]{extarticle}
\usepackage[margin=1.5cm]{geometry}
\usepackage{amsmath, amssymb}
\usepackage{array}
\usepackage{multicol}
\usepackage{helvet}
\renewcommand{\familydefault}{\sfdefault}

\begin{document}

\begin{enumerate}
    \item Resuelve las ecuaciones de primer grado
    \begin{enumerate}
        \item $0 = 4x - 3 - x + 1 - 3x + 2 $  (infinitas soluciones)
        \item $6x - 15 + 3x = x - 8 + 8x + 1$ (No hay solución)
        \item $5 - 7x + 2 - 6x = 10x - 7 - 2x$ (x=2/3)
        \item $8 + (5x - 6) = 3x - (x + 4)$ (x=-2)
        \item $10[2x - (x - 1)] + 3 = 8x - 5(x + 3)$ (x=-4)
        \item $3x - 5[1 - 3(2x + 4)] = 3[1 - 4(x - 1)]$ (x= -8/9)
        \item $\dfrac{-x-1}{6}-\dfrac{3(x+5)}{12} =  \dfrac{2(11-x)}{9}-6$ (x=11) 
        \item $x+ \dfrac{3(x-2)}{9}=\dfrac{5(x-1)}{4}+\dfrac{7}{12}$ (x=0)
    \end{enumerate}
    \item Problemas con ecuaciones de primer grado
    \begin{enumerate}
        \item Moliendo juntas dos clases de café, la primera de 7,50 €/kg y la segunda de 5,70 €/kg, se han obtenido 90 kg de mezcla que sale a 6,50 €/kg. ¿Cuánto café de cada clase se ha utilizado en la mezcla? (40 kg de café caro con 50 kg de café barato)
        \item El dinero que tiene Carlos es el cuádruple del que tiene Daniel. Si Carlos le diera 300 € a Daniel, entonces ambos tendrían la misma cantidad. ¿Cuánto dinero tiene cada uno? (Daniel 200€ y Carlos 800€)
        \item Por tres cafés y dos cruasanes hemos pagado 7,70 €. ¿Cuál es el precio de un cruasán, sabiendo que cuesta 60 céntimos menos que un café?Por tres cafés y dos cruasanes hemos pagado 7,70 €.  Cuál es el precio de un cruasán, sabiendo que cuesta 60 céntimos menos que un café? (un café cuesta 1,78 €. Un cruasán cuesta 1,18 €.)
        \item Una finca rectangular es 40 metros más larga que ancha. Al urbanizar la zona, se le recortan 8 m a lo largo y 5 m a lo ancho. Así, su perímetro se reduce en una décima parte. ¿Cuáles eran las dimensiones primitivas de la finca? (45 m de ancho y 85 m de largo)
        \item Laura tiene el triple de dinero que Marta. Si Laura le diera 400 € a Marta, entonces Laura tendría el doble que Marta. ¿Cuánto dinero tiene cada una? (Marta 1200€, Laura 3600)
        \item Aumentando un número en un 20\% y restándole dos unidades, se obtiene el mismo resultado que sumándole su séptima parte. ¿Qué número es? (35)
    \end{enumerate}
    \item Resuelve las ecuaciones de segundo grado.
    \begin{enumerate}
        \item $3x (2 - x) - 2 = 4x (x - 1) + x^2$(x=1 y x= 1/4)
        \item $16 - 5x (2x - 3) = x - 2x (3x - 1)$ (x=4 y x=-1)
        \item $15 - (x + 2)^2 = (x - 3)^2 + 2x$ (x=1 y x=-1)
        \item $\dfrac{x(x+3)}{2}-\dfrac{(x+1)^2)}{3}+\dfrac{1}{3}=0$  (x=0, x=-5)
        \item $\dfrac{(2x-3)^2}{9}+\dfrac{x}{2}= \dfrac{(x-1)^2+5}{6}$ (x=0 y x=9/5)
        \item $x(2x+1)-\dfrac{(x-1)^2}{2} = 3$ (x=1 y x=-7/3)
    \end{enumerate}
    \item Problemas con ecuaciones de segundo grado
    \begin{enumerate}
        \item El producto de mi edad actual por la edad que tenía el año pasado es igual al producto de la edad que tenía hace cuatro años por la edad que tendré dentro de cinco años. ¿Cuántos años tengo? (10 años)
        \item El producto de dos números naturales consecutivos es 90. ¿Qué números son? (-10 no es válida. Los números son 9 y 10.)
        \item La superficie de un rectángulo es 150 cm2, y su perímetro, 50 cm. ¿Cuáles son sus dimensiones? (10 cm y 15 cm)
        \item Si el lado de un cuadrado aumenta 2 cm, su superficie aumenta 28 cm2. ¿Cuánto mide el lado? (6cm)
        \item Si multiplico la edad que tenía hace dos años por la edad que tendré dentro de cuatro años, obtengo el mismo resultado que al multiplicar mi edad actual por la edad que tendré dentro de un año. (8 años)
    \end{enumerate}
    \item Otro tipo de ecuaciones
    \begin{enumerate}
        \item $x(x^2-64)=0$ (x=0, x=8 y x=-8)
        \item $(2x+1)(x^2+x-2)=0$ (x=-1/2, x=3 y x=-8)
        \item $(x-4)(\dfrac{4}{3x-1}-2) =0$ (x=4 y x=1)
        \item $\dfrac{8}{x}-3=\dfrac{5}{x+3}$ (x=-4, y x=2) Hay que comprobar las soluciones 
        \item $\dfrac{15}{x-1} = \dfrac{12}{x}+1 $ (x=6, x=-2) Hay que comprobar las soluciones 
        \item $\dfrac{7}{x+2}+2=\dfrac{9}{x-2}$ (X=5 y x=-4) Hay que comprobar las soluciones 
        \item $\dfrac{2}{3x-1}+x=\dfrac{x+3}{3x-1}$ (x=1) Hay que comprobar las soluciones 
        \item $3\sqrt{x-1} = 2x-11$ (x=10) Hay que comprobar las soluciones 
        \item $\sqrt{2x^2-2}=1-x$ (x=-3 y x=1) Hay que comprobar las soluciones 
        \item $\sqrt{4x+5} = x+2$ (x=1 y x=-1) Hay que comprobar las soluciones 
    \end{enumerate}
\end{enumerate}

\end{document}