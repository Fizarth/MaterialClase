\documentclass[a4paper,12pt]{article}
\usepackage[margin=1.5cm]{geometry}
\usepackage{amsmath, amssymb}
\usepackage{array}
\usepackage{multicol}

\begin{document}


{\Large Ejercicios de repaso de Álgebra.}

\begin{enumerate}
    \item Descompón factorialmente estos polinomios:
    \begin{multicols}{2}
        \begin{enumerate}
            \item $x^6-9x^5+24x^4-20x^3$
            \item $x^6+6x^5+9x^4-x^2-6x-9$
        \end{enumerate}
    \end{multicols}
    \item Resolver las siguientes ecuaciones:
    \begin{multicols}{2}
        \begin{enumerate}
            \item $5x-7\dfrac{-2}{3}+11\dfrac{-2}{3}=4 $ \textbf{ (x=4/3)}
            \item $\dfrac{x-2}{3}+\dfrac{2x+1}{4} = 3-\dfrac{2x-3}{6}$ \textbf{ (x=47/14)}
            \item $3x^2-15=0$ \textbf{ (x=$ \pm \sqrt{5}$)}
            \item $5x^2+7x=0$ \textbf{( x=0, -7/5)}
            \item $\dfrac{x^2+2}{x^2-6}=\dfrac{2x^2-23}{21-x^2} $ \textbf{(x=$\pm \sqrt{2},\pm$4)}
            \item $x^4+5x^2-36=0$ \textbf{(x=$\pm\sqrt{-9}$ no real $\pm 2$)}
            \item $27x^4-9x^2=0$ \textbf{(x=$\pm \sqrt{1/3}$,0)}
            \item $\sqrt{3x-2}-4=0$ \textbf{(x=6)}
            \item $\sqrt{x+4}=3-\sqrt{x-1}$ \textbf{(x=13/9)}
            \item $\dfrac{5}{6}(x-\dfrac{1}{3})+\dfrac{4}{6}(\dfrac{x}{5}-\dfrac{1}{7})=4+\dfrac{8}{9}$ \textbf{(x=5.44)}
            \item $2+\sqrt{x-5}=13-x$ \textbf{(x=14.9)}
            \item $x^4-10x^2+9=0$ \textbf{(x=$\pm 1, \pm 3$)}
            \item $x^6-9x^3+8=0$ \textbf{(x=1,2)}
            \item $\sqrt{5x+1}=5-\sqrt{x-2}$ \textbf{(x=3)}
            \item $9^x+10\cdot 3^x+9=0$ \textbf{(x=0,2)}
            \item $log(x+3)+log(x)=1$ \textbf{(x=2)}
            \item $4 log(x)+1=log(16)+log(5x)$ \textbf{(x=2)}
            \item $4^x-10 \cdot 2^x +16 = 0$ \textbf{(x=3,1)}
            \item $5^{x-1}+5^x+5^{x+1}=31$ \textbf{(x=1)}
        \end{enumerate}        
    \end{multicols}
    \item Resolver los siguientes sistemas:
    \begin{multicols}{2}
        \begin{enumerate}
            \item $\left \{
                    \begin{array}{rr}
                    2x+y-3z & = 1 \\
                    x-2y+4z & = 19 \\
                    3x+4y-z & = 1
                    \end{array}
                    \right .$   \\ \textbf{(x=-3, y=4, z=2)}
            \item $\left \{
                    \begin{array}{rr}
                    x+y-z & = 1 \\
                    2x-3y+z & = 13 \\
                    -3x+2y+5z & = -8
                    \end{array}
                    \right .$   \\ \textbf{(x=3, y=-2, z=1)}

            \item $\left \{
                    \begin{array}{rr}
                    x-2y & = 0 \\
                    x^2+y^2 & = 20
                    \end{array}
                    \right .$   \\ \textbf{(x=4, y =2 - x=-4, y=-2)}
            \item $\left \{
                    \begin{array}{rr}
                    y= & = \dfrac{2}{x-3 }\\
                    x-y & = 2
                    \end{array}
                    \right .$   \\ \textbf{(x=1, y=-1 - x=4, y=2)}
        \end{enumerate}
    \end{multicols}
    \item Resolver las siguientes inecuaciones:
    \begin{multicols}{2}
        \begin{enumerate}
            \item $4x+5<3x+7$ \textbf{(x<2)}
            \item $2(x-3)+1>5x+4$ \textbf{(x<-3)}
            \item $5(2x-1)-7x \leq \dfrac{x+5}{2}$ \textbf{(x $\leq$ 3)}
            \item $\dfrac{3x+4}{2}-4\geq \dfrac{7x-6}{3}$ \textbf{(x $\leq$ 0)}
            \item $|x-2|<3$ \textbf{(x $\in$ (-1,5))}
            \item $|2x+5|  \leq 3$ \textbf{(x $\in$ [-4,-1])}
            \item $x^3-2x^2-5x+6 \geq 0$ \textbf{(x $\in$ [-2,1] $\cup$ [3,+$\infty$])}
            \item $\dfrac{x^2-30x}{x-1} >0$ \textbf{(x $\in$ (0,1) $\cup$ (3,+$\infty$))}
            \item $\dfrac{x^2-2x+1}{x^2+x-6} \geq 0$ \textbf{(x $\in$ (-$\infty$,-3) $\cup$ \{1\} $\cup$ (2,+$\infty$))}
        \end{enumerate}
        
    \end{multicols}
    \item Resuelve los siguientes sistemas de inecuaciones:
    \begin{multicols}{2}
        \item $\left \{
                    \begin{array}{rr}
                    x+y & \leq 4\\
                    3x+8 & \leq 6
                    \end{array}
                    \right .$  
        \item $\left \{
                    \begin{array}{rr}
                    2x+y & < 4\\
                    x-3y & \geq 9
                    \end{array}
                    \right .$  
    \end{multicols}
    \item Tengo un montón de chirimoyas y unas cuantas cajas. Si meto 8 chirimoyas en cada caja me sobran 27 chirimoyas. Pero si meto 11 chirimoyas en cada caja, me sobran tres cajas. ¿Cuántas cajas tengo? \textbf{(20 cajas)}
    \item El producto de dos números naturales consecutivos es 182. Hallarlos.
    \item Halla un número sabiendo que dicho número más su mitad y menos su sexta parte es igual a 16. \textbf{(12)}
    \item Halla un número, sabiendo que el número menos la raíz cuadrada de dicho número al cuadrado menos 7 unidades es igual a uno. \textbf{(4)}
    \item El perímetro de un cuadrado es menor o igual que 25m. Calcula cuánto puede medir el lado. \textbf{(0,25/4] }
    \item Halla los lados de un rectángulo sabiendo que el perímetro mide 120m, y que la base es los 2/3 de la altura. \textbf{(b=24m, a=36m)}
    \item Calcula tres números tales que la suma de los tres es 9. El mediano disminuido en una cantidad es la tercera parte de la suma del mayor y el menor. La diferencia entre el mayor el menor excede en uno al mediano. \textbf{(x=1, y=3, z=5)}
    \item Un comerciante desea comprar dos tipos de televisores T1 y T2, que cuestan 200€ y 400€, respectivamente. Solo dispone de sitio para almacenar 20 televisores y de 5000€ para gastar. Representa el plano del recinto de todas las posibles soluciones de la cantidad de televisores de cada tipo que puede comprar.
    \item En una tienda de deportes, 2 chándales y 3 pares de deportivos cuestan 216€, y he pagado por ellos 144€. Siu en cada chándal hacen el 20\% de descuento y los deportivos el 40\%. ¿Cuánto costaba cada artículo? \textbf{(c=36€ y d=48€)}
    \item Se ha comprado un libro y un disco que costaban 48€. Sobre el precio se ha hecho una rebaja en cada artículo del 15\% y del 10\% respectivamente. Si se ha ahorrado 5.7€ ¿Cuánto costaba cada producto? \textbf{l=18€ y d=30€}
    \item Una parcela produce tres cereales diferentes: maíz, trigo y centeno. En la parcela trabajan tres agricultores durante exactamente 8 horas diarias cada uno, y se utiliza el sistema de riego durante exactamente 60 minutos. Para cuidar el maíz se emplean 2 horas de mano de obra y se necesitan 6 minutos de riego; para cuidar el trigo se emplean 4 horas de mano de obra y 4 minutos de riego; y para el centeno se emplea 1 hora de mano de obra y 4 minutos de riego. Si se deben producir exactamente 12 kilogramos en total de cereal al día por limitaciones en la producción, calcular los kilogramos de cada tipo de cereal que se producen en la parcela. \textbf{(6kg maiz, 2kg de trigo, 4kg de centeno)}
\end{enumerate}

\end{document}