\documentclass[a4paper,12pt]{extarticle}
\usepackage[margin=1.5cm]{geometry}
\usepackage{amsmath, amssymb}
\usepackage{array}
\usepackage{multicol}
\usepackage{helvet}
\usepackage{multirow}
\usepackage{graphicx}
\renewcommand{\baselinestretch}{2}
\renewcommand{\familydefault}{\sfdefault}

\begin{document}

    \begin{tabular}{| c | c | c | c |}
        \hline
        & MATEMÁTICAS & 27/02/2026 & calificación \\
        \hline 

        \multirow{2}{*}{\includegraphics{../../img-Logos/iesoGalileo.jpg}} & Nombre y apellidos \hspace{120pt} & Ex. 2º ESO UD5:  & \\ 
        &&Ecuaciones& \\ 
        \hline
    \end{tabular}
    
    \begin{tabular}{| l | }
        \hline
        NOTAS: \\ 
            1. El examen tiene que ser limpio, ordenado y sin faltas de ortografía. \\
            
            2. El examen ha de realizarse en bolígrafo negro o azul, evitando tachones en la medida \\
            de lo posible. \\ 
            
            3. Deben aparecer todas las operaciones NO vale con indicar solo el resultado.\\
            
            4. Los problemas deben contener: Datos, planteamiento y resolución, respondiendo a lo \\
            que se pregunte, NO vale con indicar un número como solución del problema.\\

            5. Persona que se vea copiando o hablando con un compañero tiene un cero en el examen\\

            6. NO está permitido el uso de calculadoras \\
        \hline
    \end{tabular}

    \begin{enumerate}
        \item (1 punto) Determina los miembros, los términos y el grado de la ecuación.
             \[7x-3 = 3x^2 +9\]
                    \begin{itemize}
                        \item miembros: 
                        \item términos:
                        \item grados: 
                    \end{itemize}
            
        \item (2 puntos) Resuelve la ecuación de primer grado:
            \[x-(x+3)-2(x+5)=5-4(x+3)\]
            $\vspace{5cm}$ 
        \item (1 punto) Resuelve la ecuación de segundo grado: 
            \[x^2+x-6=0\]
        \item (1 punto) Resuelve la ecuación de segundo grado incompleta
            \[4x^2-2x=0\]
        \item (1 punto) Resuelve la ecuación de segundo grado incompleta 
            \[2x^2-50=0\]
        \item (2 puntos) edad de un padre es cinco veces la del hijo. Si dentro de dos años la edad del padre será cuatro veces la del hijo, ¿cuál es la edad actual de cada uno? $\vspace{4cm}$ 
        \item (2 puntos) La suma de tres números pares consecutivos es 60. Calcula dichos números. $\vspace{4cm}$ 
        \item (PUNTO EXTRA) Calcula las dimensiones de una finca rectangular que tiene $12m$ más de largo que de ancho, y una superficie de 640$m^2$
    \end{enumerate}
\end{document}