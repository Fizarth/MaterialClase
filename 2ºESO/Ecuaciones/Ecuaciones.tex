\documentclass[a4paper,12pt]{extarticle}
\usepackage[margin=1.5cm]{geometry}
\usepackage{amsmath, amssymb}
\usepackage{array}
\usepackage{multicol}
\usepackage{helvet}
\usepackage{multirow}
\usepackage{graphicx}
\usepackage{hyperref} 


\begin{document}

    \subsubsection*{1- Determina los miembros, términos y el grado de las ecuaciones.}
        \begin{enumerate}
            \item $8x-1=15$
            \item $3(x-1)=2(x+1)$
            \item $x^2-4x=x^2+8$
            \item $2y+\dfrac{y}{5}=8$
            \item $t^2-5t+6=0$
            \item $a^3-3a=5a^2$
        \end{enumerate}

    \subsubsection*{2-Resolver ecuaciones de primer grado}

        \begin{enumerate}
            \item $5(x-3)-(2x+1)=4(x-1)-1$ (x=-11)
            \item $-(x+3)-2(x+5)=5-4(x+3)$ (x=6)
            \item $\dfrac{x}{6}-\dfrac{3x-1}{4}=2x+\dfrac{33}{8}$ (x=3/5)
            \item $2(x-3)+10x=\dfrac{8x-1}{2}$ (x=11/16)
            \item $\dfrac{x-1}{3}-1 =\dfrac{x+1}{6}-\dfrac{x}{2}$ (x=9/4)
        \end{enumerate}

    \subsubsection*{3-Resolver las ecuaciones de segundo grado}
        \begin{enumerate}
            \item $4x^2-13x+3=0 $ (x=1/4 y x=3)
            \item $(x + 1) (x - 1) = 2(x + 5) + 4$ (x=5 y -3)
            \item $4x^2 - x = 0$ (x=0 y 1/4)
            \item $3x^2 = 4x$ (x=0 y 4/3)
            \item $4x^2 - 1 = 0$ (x=1/2 y -1/2)
            \item $4x2 - 25 = 0$ (z=5/2 y -5/2)
        \end{enumerate}
    
    \subsubsection*{4-Resolver los problemas}

        \begin{enumerate}
            \item  La base de un rectángulo mide 9 cm más que la altura. Si su perímetro mide 74 cm, ¿cuáles serán las dimensiones del rectángulo? (altura 14cm y base 23cm)
            \item Se mezclan café natural de 7,4 € el kilo y café torrefacto de 6,8 € el kilo, y se obtienen 150 kg a 7,04 € el kilo. ¿Cuántos kilos de cada tipo de café se han mezclado? (natural 60kg y torrefacto 90kg)
            \item Una madre tiene 35 años más que su hijo, y dentro de 15 años su edad será el doble de la del hijo. ¿Cuántos años tienen en la actualidad? (hijo 20  y madre 55).
            \item  Calcula dos números enteros consecutivos cuyo producto sea 420 (20 y 21 o -21 y -20)
            \item  Calcula las dimensiones de una finca rectangular que tiene 12 dam más de largo que de ancho, y una superficie de 640 $dam^2$ (32 dam x 20 dam)
            \item  La suma de tres números pares consecutivos es 60. Calcula dichos números.(18,20 y 22)
            \item En un rectángulo la base es el doble que la altura. Calcula la longitud de sus lados si su perímetro mide 72 cm (12 cm x 24 cm)
            \item  Pablo tiene 14 años, y su madre, 42. ¿Cuántos años deben transcurrir para que la edad de la madre sea el doble de la de Pablo? (14 años)
            \item Se ha mezclado aceite de girasol de 0,8 € el litro con aceite de oliva de 3,5 € el litro. Si se han obtenido 1 000 L de mezcla a 2,96 € el litro, ¿cuántos litros se han utilizado de cada clase de aceite? (girasol 200 litros y oliva 800 litros)
            \item Un rectángulo mide 5 cm más de alto que de ancho,  y su área mide 150 $cm^2$ . ¿Cuánto miden sus lados? (10cm x 15 cm)
        \end{enumerate}
    
    \subsubsection*{Videos de ayuda}

        \begin{itemize}
            \item \href{https://youtu.be/ma5viFJKKS4?si=YZQQkJcgpC1nvw-_ }{Determinar las partes de una ecuación}
            \item \href{https://youtu.be/gJ-Ms7WjLbs?si=N7HP_PUi9e5zk5AX}{Ecuaciones de primer grado con denominadores}
            \item \href{https://youtu.be/kRGwE6OKN9M?si=5JQQugmYPSAVGTYq}{Ecuaciones de primer grado con paréntesis}
            \item \href{https://www.youtube.com/live/N0ozRYJyOiY?si=9T1DfBQMt7SMLS9i}{Ecuaciones de segundo grado}
        \end{itemize}

\end{document}