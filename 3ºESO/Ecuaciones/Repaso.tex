\documentclass[a4paper,12pt]{article}
\usepackage[margin=1.5cm]{geometry}
\usepackage{amsmath, amssymb}
\usepackage{array}
\usepackage{multicol}
\usepackage{graphicx}
\usepackage{hyperref}

\begin{document}

    \section*{Tema 5: Ecuaciones de primer y segundo grado.}

        \subsection*{Contenidos del examen}

            \begin{itemize}
                \item Ecuaciones de primer grado 
                \item Ecuaciones de segundo grado: completas e incompletas
                \item Ecuaciones bicuadradas
                \item Ruffini
                \item Ecuaciones factorizadas 
                \item Problemas 
            \end{itemize}

    \subsubsection*{1- Escribe la ecuación que expresa estos enunciados y halla el número al que hace referencia.}
            \begin{enumerate}
                \item El doble de un número más su cuarta parte es igual a 135 (60)
                \item La mitad de la diferencia de un número menos 8 unidades es 3 (14)
                \item El producto de un número por el doble de ese mismo número da como resultado (+16 y -16)
                \item El doble de la suma de un número más 4 unidades es -12 (-10)
                \item La tercera parte de un número menos la mitad de ese mismo número da como resultado -3 (18)
                \item El producto de un número por el número resultante de sumarle 5 unidades a ese mismo número es 14 (2 y -7)
                \item La cuarta parte del cuadrado de un número es iguala la tercera parte de la suma de ese número más 1 unidad. (2 y -2/3)
            \end{enumerate}

    \subsubsection*{2- Resuelve las ecuaciones de primer grado}
            \begin{enumerate}
                \item $3x-\dfrac{x+3}{4}=13$ (x=5)
                \item $4-\dfrac{x+2}{2}=x-4$ (x=6)
                \item $\dfrac{x-4}{8}+\dfrac{9-x}{12}-\dfrac{2x-7}{24}+5=x-8$ (x=13)
                \item $x+\dfrac{9(5+x)}{5}=9-x$ (x=0)
            \end{enumerate}
    \subsubsection*{3- Resuelve las ecuaciones de segundo grado}
            \begin{enumerate}
                \item $(2x+4)(x-1)+(3x+5)^2=3(2x+5)^2+x$ (x= -2 y -27)
                \item $3x^2-2(x+5)=(x+3)^2 -19$ (x=0 y 4)
                \item $3x(x+1)-\dfrac{x-2)^2}{2}=(x+1)(x-1)+15$ (x=2 y $\dfrac{-16}{3}$)
                \item $\dfrac{(x+1)^2}{2}-\dfrac{3(x-1)}{4}+\dfrac{3x(x+1)}{2}=\dfrac{3}{2}$ (x=$\dfrac{1}{8}$ y -1)
            \end{enumerate}
    \subsubsection*{4- Resuelve las ecuaciones bicuadradas}
            \begin{enumerate}
                \item $4x^4-5x^2+1=0$ (+1, -1, -1/2 y 1/2)
                \item $x^4-18x^2+81=0$ (3 y -3)
            \end{enumerate}
    \subsubsection*{5- Resuelve las ecuaciones factorizadas}
            \begin{enumerate}
                \item $x(x-1)(x-2)=0$ (0,1,2)
                \item $(x-4)(x+5)(x-3)x^2=0$ (4,-5,3,0)
            \end{enumerate}

    \subsubsection*{6- Resuelve estas ecuaciones por Ruffini}
            \begin{enumerate}
                \item $x^4 +3x^3-3x^2-11x-6$ (-1,-3,2)
                \item $x^4+6x^3+9x^2-4x-12$ (-3, -2, 1)
                \item $x^4 +x^3-19x^2-49x-30$ (-1,-2,5,-3)
            \end{enumerate}
    \subsection*{Problemas}

        \begin{enumerate}
            \item Para enlosar un salón de $48m^2$ de área se han utilizado $375$ baldosas rectangulares, en las que un lado mide cm menos que el otro. Halla las dimensiones de las baldosas. (0,4 m x 0,32m)
            \item Con una cuerda de 24m de longitud hacemos un triángulo rectángulo en el que uno de los catetos mide 6m ¿Cuánto medirán el otro cateto y la hipotenusa? (6 y 8m, hipotenusa 10m)
            \item  Tres amigos cobran 540€ por hacer un trabajo. El primero trabajó 12 horas y el segundo, que trabajó 2h más que el tercero, recibió 180€. ¿Cuántas horas y cuánto dinero corresponde a cada uno? (216€ y 144€)
            \item La edad de Rubén es la quinta parte de la edad de su padre. Dentro de 3 años, la edad de Rubén será la cuarta parte de la edad de su padre. ¿Qué edad tiene cada uno actualmente? (9 años y 45 años)
            \item Calcula dos números naturales consecutivos tales que su producto sea 132.(11 y 12)
            \item La suma de un número y su cuadrado es 42. ¿De qué número se trata? (6 y -7)
            \item El producto de un número por el doble de ese mismo número es 288. ¿Qué número es? ¿Existe más de una solución? (+12 y -12)
            \item Claudia y su madre se llevan 26 años. ¿Cuántos años tienen ahora si dentro de 10 años la edad de la madre será el triple de la edad de Claudia? (3 y 29)
        \end{enumerate}

\end{document}