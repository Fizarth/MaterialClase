\documentclass[a4paper,12pt]{article}
\usepackage[margin=1.5cm]{geometry}
\usepackage{amsmath, amssymb}
\usepackage{array}
\usepackage{multicol}
\usepackage{graphicx}
\usepackage{hyperref}

\begin{document}

 
    \Large{Nombre:}   

    \begin{enumerate}

        \item Escribe la ecuación y resuelve: 
    
        La cuarta parte del cuadrado de un número es iguala la tercera parte de la suma de ese número más 1 unidad. 
        \item Resuelve las ecuaciones:
        \begin{enumerate}
            \item \[\dfrac{x}{2}-\dfrac{2(x+2)}{7}=\dfrac{x-3}{4}\] (x=5)
            \item \[(x+1)^2+(1-x)(x+2)=0\]
            \item \[2x^2(x-7)(x+2)=0\]
            \item \[x^4+5x^2+6=0\]
        \end{enumerate}
                

        \item Reparte el número 20 en dos partes, de forma que la suma de sus cuadrados valga 202.
        
        \item La base de un rectángulo mide 5 cm más que la altura. Si disminuimos la altura en 2 cm, el área del nuevo rectángulo será de 60 cm2. ¿Cuánto miden los lados del rectángulo?
    \end{enumerate}



\end{document}