\documentclass[a4paper,12pt]{extarticle}
\usepackage[margin=1.5cm]{geometry}
\usepackage{amsmath, amssymb}
\usepackage{array}
\usepackage{multicol}
\usepackage{helvet}
\usepackage{multirow}
\usepackage{graphicx}
\renewcommand{\baselinestretch}{2}
\renewcommand{\familydefault}{\sfdefault}

\begin{document}


    \begin{tabular}{| c | c | c | c |}
        \hline
        & MATEMÁTICAS & 27/02/2026 & calificación \\
        \hline 

        \multirow{2}{*}{\includegraphics{../../img-Logos/iesoGalileo.jpg}} & Nombre y apellidos \hspace{120pt} & Ex. 3ºESO-D UD5:  & \\ 
        &&Ecuaciones& \\ 
        \hline
    \end{tabular}
    
    \begin{tabular}{| l | }
        \hline
        NOTAS: \\ 
            1. El examen tiene que ser limpio, ordenado y sin faltas de ortografía. \\
            
            2. El examen ha de realizarse en bolígrafo negro o azul, evitando tachones en la medida \\
            de lo posible. \\ 
            
            3. Deben aparecer todas las operaciones NO vale con indicar solo el resultado.\\
            
            4. Los problemas deben contener: Datos, planteamiento y resolución, respondiendo a lo \\
            que se pregunte, NO vale con indicar un número como solución del problema.\\
        \hline
    \end{tabular}


    \begin{enumerate}

        \item Escribe la ecuación y resuelve: 
    
        (1punto) La cuarta parte del cuadrado de un número es iguala la tercera parte de la suma de ese número más 1 unidad.
        $\vspace{5cm}$ 
        \item Resuelve las ecuaciones:
        \begin{enumerate}
            \item (1 punto) \[\dfrac{x}{2}-\dfrac{2(x+2)}{7}=\dfrac{x-3}{4}\] $\vspace{5cm}$ 
            \item (2 puntos) \[(x+1)^2+(1-x)(x+2)=0\] $\vspace{5cm}$ 
            \item (1 punto) \[2x^2(x-7)(x+2)=0\] $\vspace{5cm}$ 
            \item (1 punto) \[x^4+5x^2+6=0\] $\vspace{7cm}$ 
        \end{enumerate}
                

        \item (2 puntos) Reparte el número 20 en dos partes, de forma que la suma de sus cuadrados valga 202. $\vspace{7cm}$ 
        
        \item (2 puntos) La base de un rectángulo mide 5 cm más que la altura. Si disminuimos la altura en 2 cm, el área del nuevo rectángulo será de 60 cm2. ¿Cuánto miden los lados del rectángulo? $\vspace{5cm}$ 
    \end{enumerate}



\end{document}